% Options for packages loaded elsewhere
\PassOptionsToPackage{unicode}{hyperref}
\PassOptionsToPackage{hyphens}{url}
\documentclass[
]{article}
\usepackage{xcolor}
\usepackage[margin=1in]{geometry}
\usepackage{amsmath,amssymb}
\setcounter{secnumdepth}{-\maxdimen} % remove section numbering
\usepackage{iftex}
\ifPDFTeX
  \usepackage[T1]{fontenc}
  \usepackage[utf8]{inputenc}
  \usepackage{textcomp} % provide euro and other symbols
\else % if luatex or xetex
  \usepackage{unicode-math} % this also loads fontspec
  \defaultfontfeatures{Scale=MatchLowercase}
  \defaultfontfeatures[\rmfamily]{Ligatures=TeX,Scale=1}
\fi
\usepackage{lmodern}
\ifPDFTeX\else
  % xetex/luatex font selection
\fi
% Use upquote if available, for straight quotes in verbatim environments
\IfFileExists{upquote.sty}{\usepackage{upquote}}{}
\IfFileExists{microtype.sty}{% use microtype if available
  \usepackage[]{microtype}
  \UseMicrotypeSet[protrusion]{basicmath} % disable protrusion for tt fonts
}{}
\makeatletter
\@ifundefined{KOMAClassName}{% if non-KOMA class
  \IfFileExists{parskip.sty}{%
    \usepackage{parskip}
  }{% else
    \setlength{\parindent}{0pt}
    \setlength{\parskip}{6pt plus 2pt minus 1pt}}
}{% if KOMA class
  \KOMAoptions{parskip=half}}
\makeatother
\usepackage{graphicx}
\makeatletter
\newsavebox\pandoc@box
\newcommand*\pandocbounded[1]{% scales image to fit in text height/width
  \sbox\pandoc@box{#1}%
  \Gscale@div\@tempa{\textheight}{\dimexpr\ht\pandoc@box+\dp\pandoc@box\relax}%
  \Gscale@div\@tempb{\linewidth}{\wd\pandoc@box}%
  \ifdim\@tempb\p@<\@tempa\p@\let\@tempa\@tempb\fi% select the smaller of both
  \ifdim\@tempa\p@<\p@\scalebox{\@tempa}{\usebox\pandoc@box}%
  \else\usebox{\pandoc@box}%
  \fi%
}
% Set default figure placement to htbp
\def\fps@figure{htbp}
\makeatother
% definitions for citeproc citations
\NewDocumentCommand\citeproctext{}{}
\NewDocumentCommand\citeproc{mm}{%
  \begingroup\def\citeproctext{#2}\cite{#1}\endgroup}
\makeatletter
 % allow citations to break across lines
 \let\@cite@ofmt\@firstofone
 % avoid brackets around text for \cite:
 \def\@biblabel#1{}
 \def\@cite#1#2{{#1\if@tempswa , #2\fi}}
\makeatother
\newlength{\cslhangindent}
\setlength{\cslhangindent}{1.5em}
\newlength{\csllabelwidth}
\setlength{\csllabelwidth}{3em}
\newenvironment{CSLReferences}[2] % #1 hanging-indent, #2 entry-spacing
 {\begin{list}{}{%
  \setlength{\itemindent}{0pt}
  \setlength{\leftmargin}{0pt}
  \setlength{\parsep}{0pt}
  % turn on hanging indent if param 1 is 1
  \ifodd #1
   \setlength{\leftmargin}{\cslhangindent}
   \setlength{\itemindent}{-1\cslhangindent}
  \fi
  % set entry spacing
  \setlength{\itemsep}{#2\baselineskip}}}
 {\end{list}}
\usepackage{calc}
\newcommand{\CSLBlock}[1]{\hfill\break\parbox[t]{\linewidth}{\strut\ignorespaces#1\strut}}
\newcommand{\CSLLeftMargin}[1]{\parbox[t]{\csllabelwidth}{\strut#1\strut}}
\newcommand{\CSLRightInline}[1]{\parbox[t]{\linewidth - \csllabelwidth}{\strut#1\strut}}
\newcommand{\CSLIndent}[1]{\hspace{\cslhangindent}#1}
\setlength{\emergencystretch}{3em} % prevent overfull lines
\providecommand{\tightlist}{%
  \setlength{\itemsep}{0pt}\setlength{\parskip}{0pt}}
\usepackage{booktabs}
\usepackage{longtable}
\usepackage{array}
\usepackage{multirow}
\usepackage{wrapfig}
\usepackage{float}
\usepackage{colortbl}
\usepackage{pdflscape}
\usepackage{tabu}
\usepackage{threeparttable}
\usepackage{threeparttablex}
\usepackage[normalem]{ulem}
\usepackage{makecell}
\usepackage{xcolor}
\usepackage{bookmark}
\IfFileExists{xurl.sty}{\usepackage{xurl}}{} % add URL line breaks if available
\urlstyle{same}
\hypersetup{
  pdftitle={Empirical Application},
  hidelinks,
  pdfcreator={LaTeX via pandoc}}

\title{Empirical Application}
\author{}
\date{\vspace{-2.5em}}

\begin{document}
\maketitle

To demonstrate the proposed modeling strategy and examine potential
differences between CML and ARGL estimation methods, we re-analyzed the
data from a meta-analysis reported by Carter et al. (2015). The original
meta-analysis examined a large corpus of primary studies on the ego
depletion effect, which refers to the theory that an individual's
ability to exercise self-control diminishes with repeated exertion
(Hagger et al. 2010). Carter et al. (2015) argued that the apparent
strength of ego-depletion effects may be overstated due to selective
reporting. Their review included a variety of self-control manipulation
tasks as well as a range of outcomes. Some primary studies reported
effects for multiple outcome tasks, leading to dependent effects. Effect
sizes were measured as standardized mean differences, defined so that
positive effects correspond to depletion of self-control (i.e.,
consistent with the theory of ego depletion).

To mitigate possible effects of selective reporting, Carter et al.
(2015) included many unpublished studies, so that the full meta-analysis
included 116 effects from 66 studies. For illustrative purposes, we
re-analyzed the findings from the subset of published studies only; we
also excluded a single outlying effect size estimate that was greater
than 2. This analytic sample includes 66 effect size estimates from 45
distinct studies. We conducted the analyses using R Version 4.4.3 (R
Core Team 2025).

If selective reporting were not a concern, a correlated-and-hierarchical
effects model (CHE, Pustejovsky and Tipton 2022) would be one way to
summarize the distribution of ego depletion effects. Based on a CHE
model, the overall average effect estimate was 0.46, 95\% CI {[}0.34,
0.59{]}. Alternately, Chen and Pustejovsky (2024) proposed estimating
average effect sizes using a CHE model with inverse sampling covariance
weighting (CHE-ISCW), which places relatively more weight on larger
studies (those with smaller sampling variances) and thus is less biased
by selective reporting. Applying CHE-ISCW reduces the overall effect
estimate to 0.39, 95\% CI {[}0.24, 0.54{]}. As a further point of
comparison, we estimated the overall average effect using the PET/PEESE
regression adjustment (Stanley and Doucouliagos 2014), clustering the
standard errors by study. This yielded an overall average effect of
-0.09, 95\% CI {[}-0.76, 0.58{]}. The CHE and CHE-ISCW estimates are
both positive, significant, and similar in magnitude. The PET-PEESE
estimate is negative and much smaller than the CHE and CHE-ISCW
estimates, indicating a pattern of small study effects.

We used the \texttt{selection\_model()} function from the
\texttt{metaselection} package to fit single-step and two-step selection
models (Pustejovsky and Joshi 2025). The single-step model used a
threshold at \(\alpha_1 = 0.025\); the two-step model used thresholds at
\(\alpha_1 = 0.025\) and \(\alpha_2 = 0.5\). For comparison purposes, we
estimated model parameters using both CML and ARGL and computed
cluster-robust and percentile bootstrap confidence intervals. For
bootstrapping, we used two-stage cluster bootstrap re-sampling with 1999
replicates.

\begin{table}
\centering
\caption{\label{tab:empirical}Single-step and two-step selection model parameter estimates fit to ego depletion effects data from Carter et al. (2015)}
\centering
\resizebox{\ifdim\width>\linewidth\linewidth\else\width\fi}{!}{
\begin{threeparttable}
\begin{tabular}[t]{l>{\raggedright\arraybackslash}p{5.5em}>{\raggedright\arraybackslash}p{5.5em}>{\raggedright\arraybackslash}p{5.5em}>{\raggedright\arraybackslash}p{5.5em}>{\raggedright\arraybackslash}p{5.5em}>{\raggedright\arraybackslash}p{5.5em}}
\toprule
\multicolumn{1}{c}{ } & \multicolumn{3}{c}{CML estimator} & \multicolumn{3}{c}{ARGL estimator} \\
\cmidrule(l{3pt}r{3pt}){2-4} \cmidrule(l{3pt}r{3pt}){5-7}
Parameter & Estimate (SE) & Cluster-Robust CI & Percentile Bootstrap CI & Estimate (SE) & Cluster-Robust CI & Percentile Bootstrap CI\\
\midrule
\addlinespace[0.3em]
\multicolumn{7}{l}{\textbf{One-step}}\\
\hspace{1em}$\beta$ & 0.25 (0.09) & {}[ 0.07,      0.44] & {}[ 0.05, 0.47] & 0.27 (0.07) & {}[ 0.14,      0.41] & {}[ 0.11, 0.47]\\
\hspace{1em}$\tau^2$ & 0.11 (0.04) & {}[ 0.06,      0.20] & {}[ 0.02, 0.19] & 0.11 (0.04) & {}[ 0.06,      0.21] & {}[ 0.01, 0.21]\\
\hspace{1em}$\lambda_1$ & 0.27 (0.14) & {}[ 0.10,      0.75] & {}[ 0.07, 0.87] & 0.30 (0.10) & {}[ 0.16,      0.56] & {}[ 0.05, 1.20]\\
\addlinespace[0.3em]
\multicolumn{7}{l}{\textbf{Two-step}}\\
\hspace{1em}$\beta$ & 0.23 (0.11) & {}[ 0.02,      0.44] & {}[-0.01, 0.48] & 0.25 (1.45) & {}[-2.59,      3.10] & {}[-0.05, 0.54]\\
\hspace{1em}$\tau^2$ & 0.11 (0.04) & {}[ 0.06,      0.21] & {}[ 0.03, 0.19] & 0.12 (0.33) & {}[ 0.00,     32.00] & {}[ 0.00, 0.20]\\
\hspace{1em}$\lambda_1$ & 0.27 (0.14) & {}[ 0.10,      0.74] & {}[ 0.07, 0.86] & 0.29 (1.65) & {}[ 0.00,>100] & {}[ 0.05, 1.28]\\
\hspace{1em}$\lambda_2$ & 0.22 (0.16) & {}[ 0.06,      0.90] & {}[ 0.04, 1.18] & 0.26 (2.17) & {}[ 0.00,>100] & {}[ 0.01, 3.14]\\
\bottomrule
\end{tabular}
\begin{tablenotes}
\item \textit{Note: } 
\item ARGL = augmented, reweighted gaussian likelihood; CML = composite maximum likelihood; CI = confidence interval; SE = standard error.
\end{tablenotes}
\end{threeparttable}}
\end{table}

Table @ref(tab:empirical) presents the parameter estimates from the
one-step and two-step selection models. The estimated selection
parameters are similar across the one- and two-step models and across
both estimators, all indicating that non-significant or negative effect
size estimates were less likely to be reported than statistically
significant, affirmative ones. The one-step and two-step selection model
estimates of average effect size are positive but substantially smaller
than the CHE-ISCW estimates, ranging from 0.23 to 0.27 depending on the
model specification and estimation method. In contrast to the PET/PEESE
estimate, the selection model estimates using CML are positive and
statistically distinct from zero. Thus, an analyst would reach different
conclusions about overall average effect size depending on whether they
use an unadjusted model, a step-function model, or the PET/PEESE
adjustment.

The estimates in Table @ref(tab:empirical) point towards some potential
differences between estimation methods. Generally, the CML and ARGL
parameter estimates are similar in magnitude, but the confidence
intervals based on the CML estimator are narrower than those for the
ARGL estimator. For the CML estimator, the bootstrap CIs are similar or
slightly wider than the cluster-robust CIs. These patterns suggest that
there could be differences in the performance of the estimators, as well
as differences in performance between the step-function estimators and
alternative adjustment methods such as PET/PEESE. However, these results
are based on a single empirical dataset where the true data-generating
process is unknown. To draw firmer conclusion about the comparative
performance of these methods, we conducted simulations to evaluate the
methods in terms of bias, accuracy, and confidence interval coverage
across a range of conditions.

\protect\phantomsection\label{refs}
\begin{CSLReferences}{1}{0}
\bibitem[\citeproctext]{ref-carter2015series}
Carter, Evan C, Lilly M Kofler, Daniel E Forster, and Michael E
McCullough. 2015. {``A Series of Meta-Analytic Tests of the Depletion
Effect: Self-Control Does Not Seem to Rely on a Limited Resource.''}
\emph{Journal of Experimental Psychology: General} 144 (4): 796.

\bibitem[\citeproctext]{ref-chen2024adapting}
Chen, Man, and James E. Pustejovsky. 2024. {``Adapting Methods for
Correcting Selective Reporting Bias in Meta-Analysis of Dependent Effect
Sizes.''} \url{https://doi.org/10.31222/osf.io/jq52s}.

\bibitem[\citeproctext]{ref-hagger2010ego}
Hagger, Martin S., Chantelle Wood, Chris Stiff, and Nikos L. D.
Chatzisarantis. 2010. {``Ego Depletion and the Strength Model of
Self-Control: {A} Meta-Analysis.''} \emph{Psychological Bulletin} 136
(4): 495--525. \url{https://doi.org/10.1037/a0019486}.

\bibitem[\citeproctext]{ref-metaselection}
Pustejovsky, James E., and Megha Joshi. 2025. \emph{Metaselection:
Meta-Analytic Selection Models with Cluster-Robust and Cluster-Bootstrap
Standard Errors for Dependent Effect Size Estimates}.
\url{https://github.com/jepusto/metaselection}.

\bibitem[\citeproctext]{ref-pustejovsky2022preventionscience}
Pustejovsky, James E., and Elizabeth Tipton. 2022. {``Meta-Analysis with
Robust Variance Estimation: {Expanding} the Range of Working Models.''}
\emph{Prevention Science} 23 (April): 425--38.
\url{https://doi.org/10.1016/j.jsp.2018.02.003}.

\bibitem[\citeproctext]{ref-rcoreteam}
R Core Team. 2025. \emph{R: A Language and Environment for Statistical
Computing}. Vienna, Austria: R Foundation for Statistical Computing.
\url{https://www.R-project.org/}.

\bibitem[\citeproctext]{ref-stanley2014meta}
Stanley, Tom D, and Hristos Doucouliagos. 2014. {``Meta-Regression
Approximations to Reduce Publication Selection Bias.''} \emph{Research
Synthesis Methods} 5 (1): 60--78.

\end{CSLReferences}

\end{document}
